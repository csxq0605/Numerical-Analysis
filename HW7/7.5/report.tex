\documentclass{article}
\usepackage{graphicx}
\usepackage{subcaption}
\usepackage{float}

% Language setting
% Replace `english' with e.g. `spanish' to change the document language
\usepackage[english]{babel}

% Set page size and margins
% Replace `letterpaper' with `a4paper' for UK/EU standard size
\usepackage[letterpaper,top=2cm,bottom=2cm,left=3cm,right=3cm,marginparwidth=1.75cm]{geometry}

% Useful packages
\usepackage{amsmath}
\usepackage{graphicx}
\usepackage{ctex}
\usepackage[colorlinks=true, allcolors=blue]{hyperref}

\title{HW7 报告}
\author{苏王捷 2300011075}

\begin{document}
\maketitle

\section{报告要求}

我选择了自适应Simpson积分法来计算$\pi_p$值。

Simpson积分公式使用二次多项式近似被积函数,对于区间$[a,b]$上的函数$f(x)$:

$$\int_a^b f(x) dx \approx \frac{b-a}{6}\left[f(a) + 4f\left(\frac{a+b}{2}\right) + f(b)\right]$$

这种方法对于平滑函数通常有较好的精度,但对于像题目中$[u^{1-p}+(1-u)^{1-p}]^{1/p}$这样在端点处可能趋于无穷的函数,需要更精细的处理。

因而,我们需要使用自适应Simpson积分法,使用递归方式将区间不断细分,直到满足给定的误差容限。具体实现中,我设置了一个小的正数$\delta$来避免在奇异点处计算,从而提高了数值稳定性;同时设置了一个maxDepth作为递归上限防止无限递归并保证能得到返回的数值结果。

\begin{enumerate}
\item 将积分区间$[a,b]$分为两个子区间$[a,c]$和$[c,b]$,其中$c=(a+b)/2$;
\item 分别计算这两个子区间的Simpson积分值$S_L$和$S_R$;
\item 比较$S_L + S_R$与整个区间$[a,b]$的Simpson积分值$S$的差异;
\item 如果差异小于给定的误差容限,则使用$S_L + S_R$作为结果;
\item 否则,递归地对子区间应用相同的过程,直到满足给定的误差容限或达到给定的递归上限。
\item 公式中的系数15来自于Simpson公式的误差分析。当区间减半时,Simpson公式的误差理论上会减少约16倍,所以我们使用15作为一个保守的估计。表达式$(total - whole) / 15$是一个误差校正项,可以提高积分结果的精度。
\end{enumerate}

此外,我利用了题目提示的对偶性质:当1/p + 1/q = 1时,$\pi_p$=$\pi_q$,将大于2的p值转换为较小的q值进行计算,提高了计算大p值时的稳定性和效率。

该实现将积分误差控制在非常小的范围内,保证了计算结果的准确性,满足误差小于5\%的要求。

\end{document}